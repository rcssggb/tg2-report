\chapter{Desenvolvimento \label{chap:Desenvolvimento}}

% Resumo opcional. Comentar se não usar.
% \resumodocapitulo{Resumo opcional.}


\section{Biblioteca}
\par O servidor da partida apresenta, como já mencionado, um protocolo de comunicação e sintaxe de mensagens específica. Uma biblioteca de interfaceamento é proposta com o objetivo de abstrair os detalhes de comunicação de construção de mensagens e facilitar, assim, o desenvolvimento dos programas jogadores. Esta abordagem já é comum na categoria e existem soluções de código aberto como a \textit{librcsc}, utilizada por várias equipes, usualmente atreladas ao agente base \textit{agent2d}, desenvolvidas pela equipe \textit{HELIOS}.

\par Uma biblioteca própria desenvolvida em linguagem Go é proposta como forma de modernização e diversificação da base de código utilizada pelas equipes.

\section{Agente Único}
\par Inicialmente, deseja-se realizar o treinamento de um agente único que, com as informações de seus sensores, consiga com sucesso levar a bola ao gol. Essa proposta tem como objetivo fundamentar os conhecimentos e validar a base de código para a realização de um treinamento de um time de múltiplos agentes em um estágio posterior.
\par O desenvolvimento de um agente único, inicialmente, permite adquirir o entendimento necessário para a definição do vetor de estados e ajuste das técnicas de treinamento. Além disso, busca-se uma maior agilidade na substituição e teste na estrutura do vetor de estados e no algoritmo de treinamento, uma vez que o custo computacional deste cenário é consideravelmente menor que o custo do treinamento de um time completo.

% \subsection{Vetor de Estados} % não sei se faz sentido, sendo que não temos quase nada pronto sobre isso

\section{Agentes Concorrentes}

Após validação do sistema com agente único, é interessante experimentar com treinamento adversarial de apenas 2 jogadores em formato um-contra-um. A intenção dessa etapa é experimentar com o sistema o caso adversarial, no qual há um ou mais agentes com objetivo oposto ao do agente sendo treinado.

\section{Múltiplos Agentes}

Após validar os casos de agente único e de agentes concorrentes, propõe-se um treinamento completo em jogos 11 contra 11. O objetivo é, ao final do processo, termos um time capaz de jogar contra os principais times da atualidade na categoria RoboCup Soccer Simulation 2D.

Para isso, os agentes devem ser capazes de cooperar e reagir aos movimentos da equipe oposta a fim de marcar gols e evitar os gols do adversário.

% \subsection{Vetor de Estados}


