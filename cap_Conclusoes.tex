
\chapter{Conclusões}
\label{chap:Conclusoes}

Este projeto teve como objetivo a implementação de uma plataforma para desenvolvimento de aprendizagem por reforço em futebol de robôs, mais especificamente na categoria RoboCup Soccer Simulation 2D, e a utilização de técnicas de RL para realizar treinamentos de seleção de comportamentos e ações elementares para maximizar o número de gols feitos por um agente.

Ao pesquisar sobre a comunidade e equipes participantes das edições nacionais da competição, notou-se que a biblioteca de interfaceamento com o servidor \textit{librcsc} e o time base \textit{agent2d} - ambos desenvolvidos no Japão por acadêmicos da equipe \textit{HELIOS} - são amplamente utilizados. Apesar de ser uma plataforma bastante completa, sua dominância acaba limitando a inovação na categoria, mantendo as equipes dentro da arquitetura de solução proposta no \textit{agent2d}.

A plataforma desenvolvida no decorrer deste trabalho ajuda a modernizar e diversificar a base de código utilizada pelas equipes. A plataforma desenvolvida encontra-se disponível em repositórios do GitHub: \textit{rcssggb/ggb-lib}, para a biblioteca de comunicação, e \textit{rcssggb/ggb-single}, onde se encontra a rotina de treinamento. Ambas podem ser utilizadas livremente. A plataforma se conecta via protocolo UDP ao servidor que executa a partida e abstrai a codificação e decodificação de mensagens enviadas ao servidor. Algoritmos de RL implementados no decorrer do trabalho validaram o funcionamento da plataforma.

Foram realizados três treinamentos distintos e seus resultados foram analisados conforme o retorno alcançado pelo agente. Inicialmente, o algoritmo Sarsa com comportamentos pré-programados e aproximador de funções mostrou um retorno crescente com o passar dos episódios, porém o custo computacional do treinamento mostrou-se muito alto, tornando difícil a testagem de soluções desse tipo.

A fim de aumentar o número de amostras, o estado foi discretizado e implementou-se o algoritmo \textit{Q-Learning} duplo com os mesmos comportamentos e utilizando um método tabular. O número de amostras possíveis de serem coletadas aumentou consideravelmente e o treinamento se mostrou mais estável, além de obter retornos mais altos.

Por fim, foi realizado um treinamento com ações elementares, permitindo que o agente aprendesse estratégias por si só. O treinamento demonstrou retornos menores que o anterior, porém a aprendizagem pareceu continuar por mais tempo, o que sugere que abordagens com comportamentos são capazes de aproveitar melhor quantidades maiores de amostras.

\section{Trabalhos Futuros}

A plataforma desenvolvida abre muitas possibilidades para pesquisa e desenvolvimento dentro do contexto da RoboCup Soccer Simulation 2D. É possível estudar diversas técnicas de RL a fim de cumprir a proposta da competição: desenvolver um time de futebol completo com agentes capazes de cooperar entre si e disputar contra outras equipes participantes.

Apesar da abordagem simplificada feita neste trabalho, o formato oficial da competição envolve um ambiente multiagente e adversarial que introduz diversos desafios ao aprendizado uma vez que o ambiente se torna mais dinâmico e, com isso, mais imprevisível.

Além disso, a família de técnicas de RL ideais para um cenário como a RCSS é a família métodos de gradientes de políticas, ou seja, o agente busca estimar a política em si e não a função de valor de ação $Q$. Com essas técnicas, é possível parametrizar a política de modo a cobrir todo o espaço de ações e não apenas um subconjunto discreto dele.

Por fim, em um cenário competitivo é importante manter a plataforma atualizada e capaz de abstrair todas as funcionalidades do servidor. Uma das funcionalidades faltantes, por exemplo, é a da investida, também conhecida coloquialmente como "carrinho".

As equipes acadêmicas são um ambiente propício para o aprendizado prático de diversas técnicas aprendidas durante a graduação e a Universidade de Brasília tem uma longa tradição de equipes participantes em diversas categorias de robótica, incluindo outras categorias de futebol de robôs. A organização de uma equipe para desenvolvimento utilizando a plataforma a fim de participar da RCSS criaria diversas oportunidades de aprendizado e pesquisa.

% multiagente
% expandir as funcionalidades da plataforma
% adversarial
% algoritmos que tratem ações contínuas