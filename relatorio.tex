%% LyX 2.1.4 created this file.  For more info, see http://www.lyx.org/.
%% Do not edit unless you really know what you are doing.
\documentclass[a4paper,oneside,brazil,11pt,a4paper,openright,titlepage,usenames,dvipsnames]{book}
\usepackage[utf8]{inputenc}
\usepackage[T1]{fontenc}
\usepackage{lmodern}
\setcounter{secnumdepth}{3}
\setcounter{tocdepth}{3}
\usepackage{array}
\usepackage{verbatim}
\usepackage{listings}
\usepackage{calc}
\usepackage{textcomp}
\usepackage{amssymb}
\usepackage{graphicx}
\usepackage{float}

\usepackage[colorlinks=true,citecolor=black,linkcolor=black,urlcolor=black,bookmarksopen=true]{hyperref}

\usepackage{bookmark}


\lstset{
	aboveskip=6mm,
	belowskip=6mm,
	extendedchars=false,
	escapeinside='',
	showstringspaces=false,
	columns=flexible,
	basicstyle={\small\ttfamily},
	breaklines=true,
	breakatwhitespace=true,
	tabsize=2
}

\renewcommand{\lstlistingname}{Código}% Listing -> Algorithm
\renewcommand{\lstlistlistingname}{LISTA DE CÓDIGOS}% List of Listings -> List of Algorithms

\makeatletter

%%%%%%%%%%%%%%%%%%%%%%%%%%%%%% LyX specific LaTeX commands.
\pdfpageheight\paperheight
\pdfpagewidth\paperwidth

%% Because html converters don't know tabularnewline
\providecommand{\tabularnewline}{\\}

%%%%%%%%%%%%%%%%%%%%%%%%%%%%%% User specified LaTeX commands.
% Classe alternativa, apropriada para impressão frente-verso. Inclui páginas em branco
% de forma que capítulos sempre tenham início na página à direita:
% \documentclass[11pt,a4paper,openright,titlepage]{book}

% Pacotes
\usepackage[T1]{fontenc}
\usepackage[brazilian]{babel}
\usepackage{epsfig}
\usepackage{subfig}
\usepackage{amsfonts}
\usepackage{amsmath}
\usepackage[thmmarks,amsmath]{ntheorem}%\usepackage{amsthm}
\usepackage{boxedminipage}
\usepackage{geometry}
\usepackage{theorem}
\usepackage{fancybox}
\usepackage{fancyhdr}
\usepackage{ifthen}
\usepackage{url}
\usepackage{afterpage}
\usepackage{color}
\usepackage{colortbl}
\usepackage{rotating}
\usepackage{makeidx}
\usepackage{indentfirst}
% Pacotes para adição de figuras do inkscape
\usepackage{graphicx}
\usepackage{import}


% Escolher um dos seguintes formatos:
\usepackage{ft2unb} % segue padrão de fontes do Latex

\makeindex

\makeatother

\usepackage{babel}
\begin{document}
\setcounter{secnumdepth}{3}
\setcounter{tocdepth}{2}
\pagestyle{empty}

\grau{Engenheiro de Controle e Automação}

\tipodemonografia{TRABALHO DE GRADUAÇÃO}

\begin{comment}
Título
\end{comment}


\titulolinhai{PLATAFORMA DE APRENDIZAGEM POR REFORÇO}
\titulolinhaii{APLICADA A FUTEBOL DE ROBÔS BIDIMENSIONAL}

\titulolinhaiii{}

\titulolinhaiv{}

\begin{comment}
Autores. Basta retirar o texto totalmente caso não haja um determinado
autor.
\end{comment}


\autori{Bruno Andreghetti Dantas}

\autorii{Samuel Venzi Lima Monteiro de Oliveira}

\autoriii{}

\begin{comment}
Membros da banca. Basta retirar o texto totalmente caso não haja um
determinado membro da banca.
\end{comment}


\membrodabancai{Prof. Dr. Alexandre Ricardo Soares Romariz}

\membrodabancaifuncao{Orientador}

\membrodabancaii{Prof. Dr. Adolfo Bauschpiess}

\membrodabancaiifuncao{Examinador interno}

\membrodabancaiii{Prof. Dr. João Paulo Leite}

\membrodabancaiiifuncao{Examinador interno}

\membrodabancaiv{}

\membrodabancaivfuncao{}

\membrodabancav{}

\membrodabancavfuncao{}

\begin{comment}
Data de defesa: mês e ano
\end{comment}

\mes{Maio}
\ano{2021}

\begin{comment}
Comandos para criar a capa e a página de assinaturas
\end{comment}


\capaprincipal
\capaassinaturas

\begin{comment}
Ficha Catalográfica
\end{comment}

% TODO: Incluir ficha catalográfica, acho que só depois da aprovação.
% \noindent \textbf{FICHA CATALOGRÁFICA}

\noindent %
\fbox{\begin{minipage}[t]{1\columnwidth}%
BRUNO, ANDREGHETTI DANTAS; 
SAMUEL, VENZI LIMA MONTEIRO DE OLIVEIRA
% TODO: conferir o título aqui tb
Inteligência computacional para agente único de futebol de robôs,

\medskip{}


{[}Distrito Federal{]} 2021.

\medskip{}


x, 101p., 297 mm (FT/UnB, Engenheiro, Controle e Automação, 2021).
Trabalho de Graduação \textendash{} Universidade de Brasília.Faculdade
de Tecnologia.

\medskip{}


1. Aprendizagem por reforço\hfill{}2.RoboCup\hfill{}

3. Futebol de robôs

\medskip{}


I. Mecatrônica/FT/UnB\hfill{}II. Conntrole e Automação\hfill{}

%
\end{minipage}}

\noindent \medskip{}


\noindent \textbf{REFERÊNCIA BIBLIOGRÁFICA}

DANTAS, B. A.; OLIVEIRA, S. V. L. M. d. (2021). Inteligência computacional para agente único de futebol de robôs. Trabalho de Graduação
em Engenharia de Controle e Automação, Publicação FT.TG-$n^{\circ}022$,
Faculdade de Tecnologia, Universidade de Brasília, Brasília, DF, 101p.

\noindent \bigskip{}


\noindent \textbf{CESSÃO DE DIREITOS}

\noindent AUTORES: Bruno Andreghetti Dantas e Samuel Venzi Lima Monteiro de Oliveira

TÍTULO DO TRABALHO DE GRADUAÇÃO: Inteligência computacional para agente único de futebol de robôs

\noindent \medskip{}


\noindent GRAU: Engenheiro\hfill{}ANO: 2021\hfill{}

\noindent \medskip{}


É concedida à Universidade de Brasília permissão para reproduzir cópias
deste Trabalho de Graduação e para emprestar ou vender tais cópias
somente para propósitos acadêmicos e científicos. O autor reserva
outros direitos de publicação e nenhuma parte desse Trabalho de Graduação
pode ser reproduzida sem autorização por escrito do autor.

\noindent \bigskip{}


\noindent \rule[0.5ex]{1\columnwidth}{1pt}

\noindent Bruno Andreghetti Dantas e Samuel Venzi Lima Monteiro de Oliveira

\noindent Rua dos Bobos, nº 0, Bairro Feliz.

\noindent 71000-000 Brasília \textendash{} DF \textendash{} Brasil.



\begin{comment}
Dedicatória
\end{comment}


\frontmatter

\begin{comment}
Texto de dedicatória do primeiro autor.
\end{comment}

\dedicatoriaautori{Dedico este trabalho aos meus pais, Chirlene e Roberto, e ao meu irmão, Pedro.}

\begin{comment}
	Texto de dedicatória do segundo autor. Caso não tenha um segundo autor,
	este texto não será mostrado
\end{comment}

\dedicatoriaautorii{Dedico este trabalho a todos aqueles que me trouxeram até aqui}

\begin{comment}
Texto de dedicatória do terceiro autor. Caso não tenha um terceiro
autor, este texto não será mostrado
\end{comment}

% \dedicatoriaautoriii{Dedicatória do autor 3}

\begin{comment}
Comando para criar a página de dedicatória
\end{comment}

\dedicatoria

\begin{comment}
Agradecimentos
\end{comment}


\begin{comment}
Texto de agradecimentos do primeiro autor.
\end{comment}


\agradecimentosautori{
	Agradeço à minha família pela presença constante apesar da distância. Sem vocês minha jornada até aqui não seria possível.
	%
	Agradeço também aos professores que me mostraram o caminho a ser trilhado e inspiraram a cada nova disciplina minha paixão pelo curso;
	%
	aos meus colegas, que compartilharam comigo tantos dias e noites de estudo e também de jogatina por todos esses anos;
	%
	aos amigos que fiz na DROID, pelas experiências que vivemos juntos no laboratório e nas competições;
	%
	aos companheiros e ex-companheiros da GoLedger, onde evoluí muito profissionalmente ao longo dos últimos anos;
	%
	ao Samuel, que dividiu comigo essa jornada que completamos aqui;
	%
	e, por fim, ao professor Romariz pela disponibilidade e atenção na orientação do trabalho.
}

\begin{comment}
Texto de agradecimentos do segundo autor. Caso não tenha um segundo
autor, este texto não será mostrado.
\end{comment}


\agradecimentosautorii{
	Agradeço à minha família pelo papel fundamental na minha formação como indivíduo, por estar sempre presente para me divertir e me apoiar e pelo incentivo constante para que eu sempre buscasse fazer o melhor em todas as áreas da minha vida. Agradeço à Fernanda, minha namorada, pelas conversas e risadas e por compartilhar os melhores momentos comigo. Agradeço aos meus amigos que tornaram essa jornada mais leve: Victor Kzam, Ricardo Moura, André Luis, Felipe Caixeta, Gabriel Assis, Vieira Neto, Igor Beduin, Raphael Barbosa, Caio Campos, Vitor Duarte, Alice Lobo e Vitor Baltieri. Agradeço, também, à equipe UnBall por todos os desafios, noites em claro e viagens de competição. Agradeço ao Bruno, meu companheiro neste trabalho, pela amizade, competência e companheirismo nessa e em várias outras empreitadas. Finalmente, agradeço ao professor Alexandre Romariz pela orientação durante o desenvolvimento deste trabalho.
}

\begin{comment}
Texto de agradecimentos do terceiro autor. Caso não tenha um terceiro
autor, este texto não será mostrado.
\end{comment}


\agradecimentosautoriii{A inclusão desta seção de agradecimentos
é opcional e fica à critério do(s) autor(es), que caso deseje(em)
inclui-la deverá(ão) utilizar este espaço, seguindo esta formatação.}

\begin{comment}
Comando para criar a página de agradecimentos
\end{comment}


\agradecimentos

\resumo{resumo}{
O cenário da aprendizagem de máquina tem crescido cada vez mais nos últimos anos.
Junto a isso, iniciativas como a RoboCup buscam incentivar a aplicação dessas técnicas fomentando um cenário competitivo de futebol de robôs.
Com a motivação de aumentar a diversidade de ferramentas dentro da categoria \textit{RoboCup Soccer Simulation 2D}, nesse trabalho foi desenvolvida uma nova plataforma de desenvolvimento e são aplicadas técnicas de aprendizagem por reforço a fim de validá-la.
Foram realizados experimentos com técnicas estabelecidas como \textit{Sarsa} e \textit{Q-Learning} duplo tendo como objetivo realizar o maior número de gols possíveis durante o período de uma partida.
As técnicas utilizadas validaram o funcionamento da plataforma desenvolvida e a utilização de comportamentos pré-programados aliado a \textit{Q-Learning} duplo obteve uma política que alcançou gols consistentemente.

\medskip{}

Palavras Chave: Aprendizagem por reforço, RoboCup, Futebol de robôs.

}\vspace*{2cm}


\resumo{Abstract}{
The machine learning field has been increasingly growing over the last few years.
Furthermore, scientific initiatives such as RoboCup seek to promote studies and applications of these techniques by nurturing a competitive environment for robot soccer.
In this project, with the motivation of expanding the diversity of tools in the RoboCup Soccer Simulation 2D category, a new development platform was created and reinforcement learning techniques were applied to validate it.
Experiments were run using established techniques, such as Sarsa and Double Q-Learning, with the goal of scoring as many goals as possible during a match period.
The techniques used validated the platform's operation and the use of pre-programmed behaviors along with Double Q-Learning resulted in a policy capable of consistently scoring goals.
	
\medskip{}

Keywords: Reinforcement learning, RoboCup, Robot soccer

}

\begin{comment}
Listas de conteúdo, figuras e tabelas
\end{comment}


\sumario
\listadefiguras
\listadetabelas
\lstlistoflistings


\begin{comment}
Lista de Símbolos
\end{comment}


%TCIDATA{LaTeXparent=0,0,these.tex}


%\chapter*{\setfontarial\mdseries LISTA DE SÍMBOLOS} % se usar ft1unb.sty, descomente esta linha



\chapter*{LISTA DE SÍMBOLOS}

% se usar ft2unb.sty, descomente esta linha

\subsection*{Símbolos latinos}

\begin{tabular}{p{0.1\textwidth}p{0.63\textwidth}>{\PreserveBacklash\raggedleft}p{0.15\textwidth}}
	$S$ & Estado \tabularnewline
	$A$ & Ação \tabularnewline
	$R$ & Recompensa \tabularnewline
	$G$ & Retorno \tabularnewline
	$v$ & Função de valor \tabularnewline
	$q$ & Função de valor da ação\tabularnewline
 \end{tabular}

\subsection*{Símbolos gregos}

\begin{tabular}{p{0.1\textwidth}p{0.63\textwidth}>{\PreserveBacklash\raggedleft}p{0.15\textwidth}}
$\pi$ & Política de tomada de ação \tabularnewline
$\gamma$ & Fator de desconto \tabularnewline
$\alpha$ & Fator de aprendizagem \tabularnewline
$\epsilon$ & Fator de exploração \tabularnewline
$\theta_j$ & Ângulo absoluto do jogador \tabularnewline
\end{tabular}

\subsection*{Subscritos}

\begin{tabular}{p{0.1\textwidth}p{0.8\textwidth}}
$t$  & Ciclo $t$ de treinamento \tabularnewline
$terminal$  & Ciclo que encerra episódio \tabularnewline
$v_\pi$ ou  $q_\pi$ & Função de esperança do retorno de acordo com a política $\pi$ \tabularnewline
$*$  & Política ótima \tabularnewline
\end{tabular}

\subsection*{Siglas}

\begin{tabular}{p{0.1\textwidth}p{0.8\textwidth}}
IA  & Inteligência artificial\tabularnewline
RL & Aprendizagem por reforço - \textit{Reinforcement learning} \tabularnewline
RCSS & \textit{RoboCup Soccer Simulation 2D}\tabularnewline
LARC & \textit{Latin American Robotics Competition} \tabularnewline
UDP & \textit{User Datagram Protocol}\tabularnewline
MDP & Processo de decisão de Markov - \textit{Markov decision process}\tabularnewline
GPI & \textit{Generalized policy iteration} \tabularnewline

\end{tabular}


\begin{comment}
Corpo Principal
\end{comment}


\mainmatter
\setcounter{page}{1}
\pagenumbering{arabic}
\pagestyle{plain}

\begin{comment}
Introdução
\end{comment}
\chapter{Introdução}
\label{chap:Intro}

% Resumo opcional. Comentar se não usar.
% \resumodocapitulo{Resumo opcional}

\par O interesse humano em criar artefatos para facilitar seu próprio trabalho ou realizar uma tarefa sem interferência remonta os tempos mais antigos. No Egito Ptolomaico, Ctesíbio (285-222 AC) descreveu um relógio d`água com a presença de um sistema de engrenagens, um indicador e o primeiro sistema de retroalimentação registrado. Por volta de 1495, Leonardo da Vinci, concebeu o projeto de um autômato mecânico de um guerreiro em armadura medieval que podia ficar em pé, sentar-se, levantar o visor e mover os braços. \cite{guarnieri2010} 
\par O estudo da união de sistemas eletromecânicos e inteligência teve começo há pelo menos 70 anos. A \textit{cibernética}, área inaugurada por Norbert Wiener na década de 1950, descreve o estudo científico de controle e comunicação no animal e na máquina. Wiener começou, então, a desenvolver sistemas que replicassem comportamentos animais.\cite{wiener1950} Somado a isso, a teoria da informação de Claude Shannon e a teoria de computação de Alan Turing abriram espaço para pesquisas que iriam desenvolver Inteligências Artificiais (IA). \cite{pamela2004} 

\section{Robótica}
\par A robótica se apoia em conhecimentos de vários campos para criar uma das áreas de estudos mais amplas da ciência. Desde a metade do século XX, a robótica vêm reunindo noções dessas áreas, e pouco a pouco as tornando partes essenciais de si: sistemas mecânicos, eletromecânicos, teoria de controle, IA e outras. As aplicações existentes são inúmeras e se renovam a todo momento. Dentre as principais, é possível citar sistemas de manufatura, robótica médica e robótica agricultural. \cite{handbook2007} 
\par Sistemas robóticos construídos para automatizar tarefas repetitivas são interessantes. Entretanto, o avanço das indústrias e aumento da complexidade das tarefas a serem realizadas criou um ambiente catalisador para o desenvolvimento de processos de tomada de decisão autonomamente. 
\par É importante notar a complexidade do problema de se desenvolver a tomada de decisão de um sistema autônomo. Tal sistema precisa mapear seu ambiente por meio de sensores, extrair um significado do seu estado atual, usá-lo para decidir uma ação e determinar se tal ação foi a melhor a ser tomada. Sensores, porém, são imprecisos e limitados fisicamente. A representação dos estados, frequentemente, não é completamente conhecida. E o processo de mapeamento de estado para ação não é trivial.
\par Neste contexto surgiu a área de estudo conhecida como \textit{aprendizagem por reforço}, que formaliza os elementos citados anteriormente para prover uma base de como agentes devem tomar ações para cumprir um objetivo pré-definido. \cite{sutton2018reinforcement}


\section{Aprendizagem por Reforço}
\par A aprendizagem por reforço (do inglês, \textit{reinforcement learning} ou RL) tem como inspiração a maneira como o aprendizado acontece com seres-humanos: interagindo com o ambiente. \cite{sutton2018reinforcement} Se uma criança está aprendendo a andar, por exemplo, ela toma certas ações no ambiente e, ainda que inconscientemente, está atenta aos resultados que essa ação causa. 
\par A teoria por trás da aprendizagem por reforço formaliza a ideia de aprender através da interação e a aplica em um contexto computacional. Os principais elementos de um sistema de RL são: uma política de decisão, um sinal de recompensa e uma função valor.  O primeiro diz respeito à decisão de qual ação se tomar a partir de uma situação, o segundo quantifica quão boa foi a ação escolhida naquele momento e o terceiro quantifica quão boa é a ação considerando o longo prazo.\cite{sutton2018reinforcement}
\par Apesar de recente, a técnica de RL já se mostrou promissora em diversas áreas, com destaque para seu uso em jogos. Em 2016, o programa AlphaGo mostrou resultados significativos ao jogar contra o campeão europeu de Go, superando-o nos 5 jogos que foram disputados. \cite{SilverHuangEtAl16nature}

Em 2018, pesquisadores do grupo OpenAI utilizaram técnicas de RL para treinar um time de 5 agentes colaborativos no jogo \textit{DotA 2}, um jogo de estratégia em tempo real onde 2 times batalham para destruir a base inimiga. O jogo provê um ambiente extremamente complexo, com espaços de estados e ações contínuos. São $20.000$ números ponto-flutuante para estados e $1.000$ ações possíveis em um dado ciclo.
% Em contrapartida os estados em um jogo de Go são codificados com 400 números e as ações com aproximadamente 250 números.
% retirei essa parte porque achei que tornou confuso se a informação seguinte era sobre Go ou DotA 2
A duração usual de uma partida é de pelo menos 1 hora. Inicialmente, foi feito um sistema do tipo um contra um (1v1) e o agente resultante deste treinamento foi capaz de derrotar jogadores profissionais. Em 2019, o sistema com 5 agentes foi capaz de derrotar um time profissional. \cite{OpenAI_dota} Um resultado dessa magnitude foi possível devido, entre outras razões, ao número altísssimo de amostras coletadas pelos agentes: 300 anos de experiência por dia para o agente singular e 180 anos por dia por agente para o time contendo 5 membros.

\section{Futebol de Robôs}
\par A ideia de robôs jogando futebol foi proposta pela primeira vez em 1992 por Alan Mackworth\cite{mackworth1993seeing}. Desde então a comunidade científica tem criado iniciativas buscando por soluções que tornem isso realidade. Desde então a comunidade científica tem criado iniciativas buscando por soluções que tornem isso realidade.
\par Uma delas é a \textit{Robot World Cup Initiative} \cite{robocup-initiative}, abreviada como \textit{RoboCup}, que teve sua primeira edição em 1997 com mais de 40 equipes distribuídas entre as diversas categorias do evento.
\par O objetivo da iniciativa, definido pela \textit{RoboCup Federation}, é que por volta da metade do século XXI, um time de robôs humanóides autônomos vençam uma partida contra os campeões da Copa do Mundo mais recente. Mesmo que o objetivo pareça ambicioso, ele guia as pesquisas e motiva o avanço no campo.
\par Atualmente, a RoboCup conta com mais de 10 categorias, incluindo robôs humanoides, robôs com rodas e simulações. Entre elas há a \textit{RoboCup Soccer Simulation 2D}, abreviada RCSS, objeto de estudo deste projeto.

\subsection{RoboCup Soccer Simulation 2D}
\par A RCSS possui grande relevância internacional, sendo uma das principais categorias disputadas na RoboCup, com equipes do mundo inteiro.
\par A categoria apresenta, também, grande relevância no cenário brasileiro.
Desde 2005, a RCSS está presente na maior competição de robótica da América Latina, a \textit{Latin American Robotics Competition}, LARC.
\par Nessa categoria, duas equipes de 11 jogadores autônomos e independentes jogam futebol em um ambiente virtual bidimensional. Um servidor é responsável por esse ambiente e possui informação absoluta sobre o estado do jogo e suas regras. Os jogadores, por sua vez, recebem dele informação incompleta e ruidosa de seus sensores virtuais, podendo executar comandos a fim de atuar sobre o estado do jogo. \cite{rcssmanual2003}

\subsection{Servidor da partida}
\label{subsec:server}
\par Um servidor que executa a partida é disponibilizado pelos organizadores da competição e este pode ser utilizado, também, para desenvolvimento. O servidor, portanto, apresenta, internamente, algumas das regras da partida bem como um juiz autônomo que age para determinar gols, faltas e demais situações de uma partida de futebol. Caso necessário, um juiz humano poderá intervir em situações não contempladas pelas regras do servidor.
\par O servidor simula todos os movimentos e ações dos jogadores e da bola. Clientes externos se conectam ao servidor e cada cliente controla um único jogador. A comunição entre o cliente e o servidor é feita a partir do protocolo UDP por meio de mensagens com sintaxe específica e definida pelo servidor.
\par De forma a permitir o acompanhamento visual da partida, um monitor também é disponibilizado, porém não é necessário para que uma partida ocorra com sucesso.
\par O servidor, ainda, possui o modo \textit{trainer} para utilização durante treinamentos de algoritmos de inteligência computacional. Este modo permite a conexão de um cliente do tipo treinador que tem acesso absoluto às informações da partida e pode mudar modos de jogo e ainda mover arbitrariamente jogadores e bola. Adicionalmente, é possível acelerar os ciclos da partida permitindo o treinamento em tempo hábil.

\begin{figure}[h]
	\includegraphics[width=0.9\linewidth]{figs/server.png}
	\centering
	\caption{Visualização de uma partida em andamento}
	\label{fig:rcssserver}
\end{figure}

\subsection{Cliente}
\par Os jogadores são controlados por clientes externos conectados ao servidor. Como já foi dito, um cliente corresponde a um único jogador e os clientes só podem ser comunicar com mensagens mandadas através do servidor da partida.
\par O cliente pode ser desenvolvido em qualquer linguagem desde que se comunique com o servidor pelo protocolo UDP e utilize a sintaxe de mensagens reconhecida pelo sistema. Há várias escolhas disponíveis para a construção do cliente, sendo decisão de cada equipe competidora como fazê-lo.

\begin{figure}[H]
	\includegraphics[width=0.9\linewidth]{figs/system.png}
	\centering
	\caption{Esquema ilustrando a arquitetura de um cliente e sua comunicação com o servidor do jogo.}
	\label{fig:system}
\end{figure}

\subsection{Sensores}
\par Cada jogador presente na partida possui um conjunto de sensores de onde são tiradas todas as informações sobre o ambiente. Em uma partida usual, um jogador tem informações visuais dos jogadores do seu time e do time adversário, da bola e de uma série de marcadores fixos no campo, como bandeiras e linhas, que servem para situar o jogador em coordenadas absolutas do campo. O jogador possui também informações ``sonoras'', onde pode ouvir mensagem do árbitro, treinador e de outros jogadores. Por último, tem acesso a informações do próprio corpo, como orientação do corpo e pescoço. \cite{rcssmanual2003}
\par Os sensores possuem características que os aproximam de sensores reais como perda de resolução da informação conforme a variável medida se afasta do sensor.


\subsection{Ações}
\label{sec:actions}

A cada ciclo de simulação, cada cliente conectado ao servidor pode realizar ações que terão efeito no ambiente.\cite{rcssmanual2003}

As ações englobam mover-se, virar-se, chutar a bola e até falar, permitindo troca de mensagens entre os jogadores. As ações disponíveis serão detalhadas no decorrer do texto.


\subsection{Abordagens utilizadas na categoria}
\label{subsec:abordagens}
\par Uma pesquisa sobre as abordagens para o desenvolvimento das estratégias dos times participantes da RCSS revelou o uso recorrente de métodos de inteligência computacional.
\par A equipe chinesa \textit{WrightEagle}, campeã do principal evento internacional da categoria diversas vezes, utiliza Processos de Decisão de Markov ou MDPs para modelar a partida\cite{bai2015online}.
\par A equipe japonesa \textit{HELIOS}, campeã de 2018 da categoria na RoboCup, divide seus jogadores em categorias ``chutadores'' e ``não-chutadores''.
Os chutadores são responsáveis por realizar o planejamento de sequência de ações, utilizando métodos de valor de ação.
Os não-chutadores, por sua vez, não tem conhecimento do planejamento feito pelos chutadores, e devem obter o máximo de informações relevantes para tentar gerar a mesma sequência de ações que jogador chutador\cite{nakashima2018helios2018}.
\par A equipe brasileira \textit{ITAndroids}, atual campeã da LARC, utiliza a abordagem de sequência de ações, similar à \textit{HELIOS}, explorando uma árvore de ações criada dinamicamente de forma a maximizar o valor de cada ação. Além disso, utilizam Otimização por Enxame de Partículas \cite{melloitandroids} para adequar os parâmetros que calculam o valor da ação. A \textit{ITAndroids} também vem desenvolvendo o uso de Aprendizagem por Reforço Profunda \cite{maximoitandroids}.
\par Muitas equipes, ainda, desenvolvem seus agentes utilizando o agente base da equipe \textit{HELIOS}, \textit{Agent2d} com a biblioteca \textit{Librcsc}, escritas em C++. Por isso, é comum que haja semelhança na construção dos agentes dessas equipes.

\section{Caracterização do Problema}
\par Deseja-se, então, explorar o problema de se fazer gols com um agente único no ambiente descrito utilizando-se de técnicas de aprendizagem por reforço.
\par Para isso, foi desenvolvida uma biblioteca de interfaceamento com o servidor da partida como adaptação do ambiente. Após isso o agente foi treinado com a utilização de técnicas de RL em duas abordagens - ações puras e comportamentos - de forma a compará-las.
\par No Capítulo 2, descreve-se os fundamentos teóricos da aprendizagem por reforço, com destaque para os algoritmos de \textit{Sarsa}, \textit{Q-learning} e \textit{Q-learning} duplo. No Capítulo 3, descreve-se o desenvolvimento da biblioteca e a definição de ações e comportamentos. Além disso, propõe-se o algoritmo de treinamento e os experimentos a serem realizados. No Capítulo 4, os resultados desses experimentos e suas análises são apresentados. Finalmente, o Capítulo 5 expõe conclusões e trabalhos futuros.

\subsection{Objetivos}
\par De acordo com o contexto apresentado, o presente trabalho se propõe a cumprir as seguintes etapas:
\begin{itemize}
	\item Implementar uma biblioteca de interfaceamento para comunicação com o servidor
	\item Utilizar técnicas de aprendizagem por reforço para treinar a escolha de comportamentos
	\item Utilizar técnicas de aprendizagem por reforço para treinar a escolha de ações puras
	\item Comparar as diferentes configurações de treinamento e seus resultados
\end{itemize}



\begin{comment}
Fundamentos
\end{comment}
\chapter{Fundamentação Teórica \label{chap:FundamentacaoMatematica}}

% Resumo opcional. Comentar se não usar.
% \resumodocapitulo{Resumo opcional.}


\section{Processos de Decisão de Markov}

O problema abordado neste trabalho pode ser descrito como um Processo de Decisão de Markov (MDP).
MDP é uma forma clássica de representação matemática de processos de decisão sequenciais.
Nessa representação, cada ação tomada por um agente que interage com o ambiente transforma o estado do processo e determina a recompensa que o agente recebe imediatamente.
Esse estado também deve ser suficiente para conter toda a informação relevante para a dinâmica futura do processo.

\begin{figure}[h]
	\includegraphics[width=0.6\linewidth]{figs/RL.png}
	\centering
	\caption{Interação agente-ambiente em um MDP \cite{sutton2018reinforcement}.} % figure 3.1 page 48
	\label{fig:mdp_env}
\end{figure}

Assim, dado um espaço de estados $\mathcal{S}$, um espaço de ações $\mathcal{A}$ e um espaço de recompensas $\mathcal{R}$, para cada par $(S, A)$ com $S \in \mathcal{S}$ sendo o estado atual do processo e $A \in \mathcal{A}$ a ação tomada pelo agente existe uma determinada probabilidade de atingir o estado $S' \in \mathcal{S}$ e receber a recompensa imediata $R \in \mathcal{R}$ \cite{sutton2018reinforcement}.

Essa abordagem é bastante flexível e torna possível a modelagem da dinâmica do futebol virtual de robôs de diversas maneiras de modo que cada agente possa construir um estado percebido a partir de seus sensores e tomar decisões acerca de qual ação tomar diante desse estado a fim de maximizar a recompensa recebida.

\subsection{MDP Episódico e Contínuo}

Um MDP pode ser caracterizado quanto à presença de um estado terminal. Caso o MDP tenha um ou mais estados que determinem o fim do processo, ele é dito episódico. A simulação de futebol de robôs tratada neste trabalho é um exemplo de MDP episódico, uma vez que o MDP termina ao se encerrar o tempo de jogo.

Em contrapartida, há MDPs onde não está bem definido nenhum estado terminal. Nesses casos, o MDP pode continuar indefinidamente até que uma ação externa ao MDP determine a sua parada. Um exemplo disso é um MDP que controle um robô numa linha de produção. Caso o sistema de automação supervisor desse robô não determine sua parada (por falta de insumos, por exemplo), o MDP pode seguir operando indefinidamente.

Nesta fundamentação, será tratado com mais atenção o caso episódico uma vez que é o caso que interessa para aplicação na simulação de futebol de robôs.

\subsection{Recompensa e Retorno}

Como definido acima, para cada ação tomada em um MDP é atribuída uma recompensa $R \in \mathcal{R}$. Essa recompensa é sempre referente ao instante de tempo anterior, ou seja, não depende de qualquer outro fator que não o par $(S_t, A_t)$ executados no instante $t$ e a função de probabilidade associada pelo MDP a esse par. Por isso, é comum utilizar a notação $R_{t+1}$ para se referir à recompensa obtida após tomar a ação $A_t$ no instante de tempo $t$.

Porém em muitos casos é esperado de um agente que ele tome decisões que maximizem a recompensa total ao fim de um episódio, ou seja, é esperado que se escolha $A_t$ a fim de maximizar não apenas $R_{t+1}$ mas sim o retorno $G_{t+1} = R_{t+1} + R_{t+2} + \dotsc + R_{terminal}$.

\subsection{Políticas}

É dado o nome de política para qualquer função $\pi(S) \to \mathcal{A}$ que leve de um estado qualquer do MDP para uma ação a ser tomada. Para cada política $\pi$, existe uma função $q_\pi(S, A)$ que, para cada par de estado e ação, define a esperança de retorno caso o agente continue seguindo a política $\pi$ no restante do episódio.

É possível comparar duas políticas $\pi$ e $\pi'$ a respeito de suas funções $q$. A política $\pi$ é considerada melhor ou igual a $\pi$, ou $\pi \ge \pi'$, caso $q_\pi(S, A) \ge q_{\pi'}(S, A)$ para todo par $(S, A)$.

Sempre há ao menos uma política melhor ou igual a todas as outras, denominada política ótima. Qualquer política que cumpra esse requisito é denominada $\pi_*$ e, caso haja mais de uma, todas devem possuir a mesma função $q$ denominada $q_*$ \cite{sutton2018reinforcement}. % chap 3.6

Uma política que toma sempre o caminho de maior retorno é denominada gulosa, e uma política que toma o caminho de maior retorno mas escolhe uma ação aleatoriamente com probabilidade parametrizada $\epsilon$ é denominada $\epsilon$-gulosa.

\section{Aprendizagem por Reforço}

Dada uma modelagem do problema como um MDP, resta obter uma maneira de estimar as probabilidades que determinam a dinâmica desse MDP, e com isso determinar um critério de decisão - denominado política - capaz de maximizar a recompensa a longo prazo recebida pelo agente.

O conjunto de técnicas que resolvem esse tipo de problema é chamado de Aprendizagem por Reforço.
No campo da aprendizagem de máquina, ela se difere da Aprendizagem Supervisionada por não haver um conjunto de pares $(s, a)$ dados como corretos.
Nesse tipo de aprendizagem, o objetivo é extrapolar uma solução genérica a partir de exemplos de um conjunto de treinamento dado como correto, o que não é prático em problemas em que não se tem exemplos de comportamentos corretos e que representem bem o conjunto total de situações possíveis.
Ela também se diferencia da Aprendizagem Não-Supervisionada, que tradicionalmente visa encontrar estrutura em conjuntos de dados não classificados, enquanto a Aprendizagem por Reforço visa maximizar um sinal de recompensa \cite{sutton2018reinforcement}.

Desse modo, as técnicas de Aprendizagem por Reforço serão aplicadas a fim de buscar políticas capazes de maximizar o desempenho dos jogadores virtuais, ou seja, obter políticas que tornem os agentes capazes de fazer gols e evitar que os jogadores do time adversário façam gols.

\subsection{Aprendizagem On-policy e Off-policy}

Entre as técnicas de aprendizagem por reforço existe uma divisão entre a aprendizagem on-policy e a aprendizagem off-policy, referentes à relação entre a política executada durante o aprendizado e a política sobre a qual se quer aprender.

Nos algoritmos de aprendizagem on-policy, o agente aprende a respeito da política $\pi$ enquanto navega o MDP de acordo com a própria política $\pi$.

Já nos algoritmos de aprendizagem off-policy, o agente aprende a respeito da política alvo $\pi$ enquanto navega o MDP de acordo com a política $b$, ou seja, ele estima a função $q_{\pi}$ enquanto segue a política $b$.

Os métodos off-policy costumam introduzir variância no processo, tornando o aprendizado ruidoso e em alguns casos a garantia de convergência é provada apenas para o caso on-policy. % TODO: carece de fonte

Além disso, é possível observar que a aprendizagem on-policy é apenas um caso particular da aprendizagem off-policy em que $b = \pi$.

\subsection{Soluções Tabulares e Aproximadas}

A maioria dos métodos de aprendizagem por reforço são testados e validados em MDPs cujos espaços de estados $\mathcal{S}$ e de ações $\mathcal{A}$ são suficientemente pequenos. Para esses MDPs é possível utilizar uma solução tabular, ou seja, a função $Q$ poda ser armazenada em uma tabela de tamanho razoável e sua imagem para cada par estado-ação pode ser atualizado individualmente.

Infelizmente, em diversas aplicações a quantidade de estados possíveis é grande demais ou até mesmo infinito, como é o caso de sistemas em que determinada característica do estado é medida como uma grandeza contínua. Nesses casos, é impossível esperar que se obtenha soluções ótimas mesmo com tempo infinito, portanto o objetivo é obter uma solução aproximada que seja boa o suficiente para a aplicação desejada.

A ferramenta matemática utilizada para viabilizar soluções aproximadas é o conceito de aproximadores de função, muito utilizados na aprendizagem supervisionada. Entre os aproximadores mais utilizados estão os aproximadores lineares e as redes neurais multicamada.

Neste trabalho serão utilizados métodos de solução aproximada devido à grande quantidade de informações simultâneas às quais o jogador tem acesso.

\subsection{Q-Learning}

Um dos algoritmos mais populares no campo da aprendizagem por reforço é o Q-Learning. Trata-se de um método off-policy que aproxima diretamente a função $q_*$ independente da política que estiver sendo adotada pelo agente durante o treinamento.

O algoritmo é também muito simples. Dada uma representação tabular $Q: (\mathcal{S},\mathcal{A}) \to \mathbb{R}$ da função $q_*$, para cada instante de tempo $t$ é realizada a seguinte atualização a fim de aproximar $Q$ de $q_*$:

\begin{equation}
Q(S_t, A_t) \leftarrow Q(S_t, A_t) + \alpha[R_{t+1} + \max_{a} Q(S_{t+1}, a) - Q(S_t, A_t)]
\end{equation}

Após iterações suficientes, espera-se que $Q$ convirja para $q_*$. Em alguns casos a convergência é provada matematicamente.

Uma vez estimada a função $q_*$, é simples obter a política ótima. Basta escolher a ação que maximiza $q_*$ no estado atual, ou seja:

\begin{equation}
A_{t+1} = \max_{a} q_*(S_t, a)
\end{equation}

É comum, mas não obrigatório, que a política $b$ seguida durante o aprendizado seja $epsilon$-gulosa em relação à aproximação Q.

\subsection{Q-Learning Duplo}

Apesar de popular o Q-Learning possui um problema de viés de maximização. Uma vez que a aproximação $Q$ é imprecisa no início do treinamento, é possível que o retorno esperado estimado seja enviesado para um valor maior do que o real.

Como solução para esse problema, é utilizada a abordagem do Q-Learning Duplo. Nela são utilizadas duas aproximações, $Q_1$ e $Q_2$, e a atualização de $Q$ é dada da seguinte forma:

\begin{equation}
\label{eq:doubleq}
Q_1(S_t, A_t) \leftarrow Q_1(S_t, A_t) + \alpha[R_{t+1} + Q_2(S_{t+1}, \text{arg}\max_a Q_1(S_{t+1}, a)) - Q_1(S_t, A_t)]
\end{equation}

Em metade das iterações (através de um sorteio, por exemplo), as aproximações $Q_1$ e $Q_2$ são trocadas. Com isso é anulado o viés de maximização gerado pelo uso de $\max_a Q$ como estimativa de retorno para os estados seguintes.

A vantagem desse método é que apesar de dobrar os requisitos de memória do algoritmo, afinal será preciso armazenar os dados referentes a duas aproximaçoes, ele não aumenta o custo computacional por iteração.


\begin{comment}
Desenvolvimento
\end{comment}
\chapter{Desenvolvimento \label{chap:Desenvolvimento}}

% Resumo opcional. Comentar se não usar.
% \resumodocapitulo{Resumo opcional.}


\section{Biblioteca} \label{sec:lib}

O servidor da partida apresenta, como já mencionado, um protocolo de comunicação e sintaxe de mensagens específica. Uma biblioteca de interfaceamento foi desenvolvida com o objetivo de abstrair os detalhes de comunicação de construção de mensagens e facilitar, assim, o desenvolvimento dos programas jogadores. Esta abordagem já é comum na categoria e existem soluções de código aberto como a \textit{librcsc}, utilizada por várias equipes, usualmente atreladas ao agente base \textit{agent2d}, desenvolvidas pela equipe \textit{HELIOS}.

A biblioteca própria foi desenvolvida em linguagem Go como forma de modernização e diversificação da base de código utilizada pelas equipes. A biblioteca cobre uma parte considerável das possibilidades previstas no protocolo de comunicação e foi programada de modo a ser facilmente expansível de acordo com o lançamento de novas versões do servidor.

\subsection{Arquitetura do código}
A biblioteca possui três pacotes internos: \textit{playerclient}, \textit{trainerclient} e \textit{rcsscommon}. Os dois primeiros dizem respeito aos dois tipos de programas que podem se conectar ao servidor da partida: jogadores e treinadores. O terceiro engloba todas as funcionalidades utilizadas por ambos clientes, além de informações gerais sobre parâmetros da partida, como coordenadas de bandeiras do campo e modos de jogo.

Os dois clientes desenvolvidos possuem as funcionalidades necessárias para se conectar ao servidor, ouvir mensagens via protocolo UDP, decodificá-las e então executar uma ação em forma de mensagem codificada e enviada ao servidor.

\subsection{Decodificação de Codificação de Mensagens}
\label{sec:messsages}
% lexer -> analisador léxico
% parser -> analisador sintático
A decodificação de mensagens, por sua vez, foi feita em duas camadas: um analisador léxico e um analisador sintático. O lexer passa pelas mensagens em formato string e retira todas as informações que ela contém. O analisador sintático estrutura essas informações em estruturas de dados para que possam ser utilizadas fora da biblioteca. As informações recebidas e decodificadas são, em sua maioria, dados dos sensores do jogador.
\begin{center}



\begin{tabular}{c}
\begin{lstlisting}
(see 37 ((f c b) 16.6 -1 -0 -0.8))
\end{lstlisting}
\end{tabular}

Exemplo de mensagem codificada.


\begin{tabular}{c}
\begin{lstlisting}
SightSymbols{
	Time: 37,
	ObjMap: map[string][]string{
		"f c b":    {"16.6", "-1", "-0", "-0.8"},
	},
}
\end{lstlisting}
\end{tabular}

Mensagem após passar pelo analisador léxico.



\begin{tabular}{c}
\begin{lstlisting}
SightData{
	Time: 37,
	Ball: nil,
	Lines: LineArray{},
	Flags: FlagArray{
		{
			ID:        rcsscommon.FlagCenterBot,
			Distance:  16.6,
			Direction: -1,
		},
	},
}
\end{lstlisting}
\end{tabular}

Mensagem após passar pelo analisador sintático.

\end{center}

\section{Definição dos Estados}
\par As informações de estado são fornecidas pelos sensores do jogador. Há três sensores presentes que entregam informações, em forma de mensagens, para o agente: sensor auditivo, sensor visual e sensor corporal.
\par Além de ser necessário tratar as mensagens recebidas, como demonstrado na seção \ref{sec:messsages}, é necessário tratar alguns dos estados para dar mais informações ao agente. 

\subsection{Sensores}
\par Os sensores são responsáveis por todas as informações que o jogador tem do ambiente. Eles são modelados de forma a emular características de sensores reais, então, dessa forma, um sensor pode ``perder'' informações caso a variável medida esteja longe, como será evidenciado no modelo do sensor visual.
\subsubsection{Sensor Auditivo}

As mensagens do sensor auditivo são do seguinte formato:

\textit{(hear Tempo Remetente "mensagem")}

Onde \textit{Tempo} é o número do ciclo em que a mensagem foi ouvida e \textit{Remetente} é descrição de quem enviou a mensagem. O \textit{Remetente} pode ser o árbitro, outros jogadores, um dos treinadores ou o próprio jogador.

No escopo deste trabalho, apenas as mensagens do árbitro serão consideradas, não sendo implementada nenhuma forma de comunicação direta entre os jogadores.

\subsubsection{Sensor Visual}
\label{sec:visual}
As mensagens do sensor visual contém as posições relativas referentes a cada objeto dentro do campo de visão do jogador. Esses objetos podem ser outros jogadores, marcadores como bandeiras e linhas (Figura \ref{fig:flags}) e a bola. O formato genérico é este:

\textit{(see (Objeto1)(Objeto2)(Objeto3)...(ObjetoN))}

Onde cada objeto tem o seguinte formato:

\textit{((NomeDoObjeto) Distância Direção VariaçãoDeDistância VariaçãoDeDireção DireçãoDoCorpo DireçãoDaCabeça)}

Sendo a distância e a direção dados em coordenadas polares, assim como suas variações. As informações de direção do corpo e da cabeça só aparecem quando o objeto em questão é um outro jogador.

\begin{figure}[H]
	\includegraphics[width=0.9\linewidth]{figs/flags.png}
	\centering
	\caption{Indicadores espalhados pelo campo para que o agente possa estimar sua posição absoluta \cite{rcssmanual2003}.}
	\label{fig:flags}
\end{figure}

A riqueza de detalhes a respeito das informações obtidas depende da distância entre o objeto e o jogador. Por exemplo, caso esteja sendo visto outro jogador a uma distância muito grande, talvez seja impossível determinar o número de sua camisa ou até mesmo a qual time ele pertence. Em contrapartida, para jogadores próximos, é fornecido até mesmo a direção para a qual ele está olhando.

\subsubsection{Sensor Corporal}

O sensor corporal contém informações sobre o estado físico do jogador. Entre elas sua energia, que é consumida a cada ação tomada como chute ou arrancada (Seção \ref{sec:actions}), sua própria velocidade e direção de movimento, a direção de sua cabeça e a quantidade de cartões de advertência recebidos.

\subsection{Estimação e Tratamento de Estados}
\par Os sensores do jogador fornecem somente informações em coordenadas polares relativas ao próprio jogador. Desta forma, ele possui informações de distância e direção para a bola, demais jogadores, bandeiras do campo e linhas do campo. 
\par Com esses dados, entretanto, é possível estimar as coordenadas cartesianas absolutas no campo de todas as entidades de interesse. As bandeiras do campo são fixas e possuem coordenadas conhecidas, seção \ref{sec:visual}. Portanto, a transformação da informação polar e relativa para uma cartesiana e absoluta da posição do próprio jogador é direta utilizando trigonometria básica. 
\par A partir da informação de posição absoluta do próprio jogador e de sua direção é possível calcular as posições para o resto das entidades.
% TODO: decidir se falar do NotSeenBallFor é ok já que a gente não treinou com ele
\par É possível, também, extrair estados úteis a partir das informações dos sensores, como por exemplo um estado que representa a quanto tempo o jogador não vê a bola. Este estado pode ser determinante para que o jogador escolha buscar a bola.

\subsection{Discretização}
% TODO
\section{Definição de Ações}
% TODO: mover detalhamento de ações para cá

\subsection{Discretização do Espaço de Ações}

\subsection{Definição de Comportamentos}

\section{Seleção de Comportamentos e Ações}

\section{Ambiente de Treinamento}

O ambiente de treinamento consiste em uma base de código que importa a biblioteca detalhada na seção \ref{sec:lib}. Um formato geral foi definido e desenvolvido a fim de tornar os experimentos fáceis de adaptar, bastando mudar alguns trechos do código.

\subsection{Laço de Treinamento}

De forma geral o laço de treinamento obedece o pseudo-código apresentado.

\begin{tabular}{c}
	\begin{lstlisting}
		definir parametros
		inicializar pesos de treinamento
		enquanto o estado nao e terminal:
			conectar jogador
			laco para cada passo da partida:
				escolher acao de acordo com a politica e o estado
				observar novo estado e recompensa
				treinar pesos
				estado <- novo estado
	\end{lstlisting}
\end{tabular}

O laço interno é onde efetivamente os algoritmos de treinamento são implementados, portanto este trecho é alterado a depender da técnica de aprendizagem por reforço utilizada.


\subsection{Experimentos}

% listar experimentos que serão apresentados a seguir
% double q tabular
% double q tabular behaviors

\begin{comment}
Resultados Experimentais
\end{comment}
\chapter{Resultados Experimentais}
\label{chap:Resultados}

% Gráficos e comentários provenientes de cada um dos experimentos.

% Resumo opcional. Comentar se não usar.
% \resumodocapitulo{Resumo opcional.}



\section{Sarsa Aproximado e Comportamentos Pré-Programados}

O treinamento utilizando o algoritmo Sarsa realizado contou com 30.000 partidas utilizando a rede neural multicamadas descrita na Subseção \ref{subsec:sarsadev} e salvando os retornos obtidos em cada partida.

O gráfico da Figura \ref{fig:single-agent-sarsa-behaviors} mostra em conjunto o retorno a cada partida e a média móvel do retorno com janela de 1000 partidas. É possível observar que há um tendência de subida do retorno com o passar dos episódios experienciados, porém em grande parte das partidas o agente não realizou nenhum gol, refletido pelo baixo valor da média de 100 partidas.

É interessante ressaltar o alto custo computacional deste tipo de treinamento devido à utilização de redes neurais. Dessa forma, a quantidade de amostras possíveis de serem coletadas em tempo hábil foram drasticamente reduzidas.

Além disso, a utilização de métodos aproximados posa um problema de duplo aprendizado: deseja-se aprender a política ótima enquanto se aprende a aproximar esta política ótima desconhecida por meio de uma rede neural. Esse fator contribui negativamente no tempo para convergência da política. Em métodos tabulares, apesar do maior custo de memória, esse problema é inexistente.

\begin{figure}[H]
	\includegraphics[width=0.9\linewidth]{figs/sarsa-tmp.png}
	\centering
	\caption{\textbf{[PLACEHOLDER]} Curva de aprendizado do agente com comportamentos pré-programados utilizando Sarsa aproximado.}
	\label{fig:single-agent-sarsa-behaviors}
\end{figure}

\section{\textit{Q-Learning} duplo Tabular e Comportamentos Pré-Programados}
\label{sec:behaviors-tabular}
Substituindo o aproximador de função por uma tabela e o algoritmo Sarsa pelo Q-Learning, foram executados 3 treinamentos de 100000 partidas e salvos a tabela Q completa e o histórico dos retornos obtidos pelo agente ao longo do treinamento.

O gráfico da Figura \ref{fig:single-agent-tabular-behaviors} mostra o histórico médio dos 3 treinamentos. Observa-se que
% o desempenho dessa abordagem supera o da abordagem anterior rapidamente, com poucas amostras. Em contrapartida, 
há uma estagnação do retorno por volta das 60000 amostras, o que pode indicar a necessidade de ajuste no decaimento do fator de exploração para que o agente explore novas possibilidades por mais partidas ou a existência de um limite superior para o desempenho do agente devido à menor flexibilidade da política aprendida, ou seja, o agente só é capaz de construir a política a partir dos comportamentos pré-programados.

\begin{figure}[H]
	\includegraphics[width=0.9\linewidth]{figs/curva-behaviors-tabular.jpg}
	\centering
	\caption{Curva de aprendizado do agente com comportamentos pré-programados.}
	\label{fig:single-agent-tabular-behaviors}
\end{figure}

A Figura \ref{fig:goal-seq} ilustra o agente conduzindo a bola e realizando um gol conforme a política aprendida.

\begin{figure}[H]
	\includegraphics[width=0.9\linewidth]{figs/goal-sequence.png}
	\centering
	\caption{Sobreposição de sequência de imagens do agente fazendo gol.}
	\label{fig:goal-seq}
\end{figure}

A Figura \ref{fig:curvalonga-bhv} mostra o histórico de retornos para um treinamento mais longo, de 200000 partidas. Nela percebe-se que após a cessação da exploração o agente estabiliza seu desempenho um pouco abaixo do máximo obtido.

\begin{figure}[H]
	\includegraphics[width=0.9\linewidth]{figs/curvalonga-behaviors-tabular.jpg}
	\centering
	\caption{Curva de aprendizado do agente com comportamentos pré-programados para treinamento longo.}
	\label{fig:curvalonga-bhv}
\end{figure}

\section{\textit{Q-Learning} duplo Tabular e Ações Puras}

Substituindo os comportamentos pré-programados por uma seleção de ações puras, foram executados 3 treinamentos distintos de 100000 partidas a fim de suavizar o elemento sorte nos resultados. Após cada um dos treinamentos foram salvos a tabela Q completa e o histórico dos retornos obtidos pelo agente ao longo do treinamento.

A Figura \ref{fig:single-agent-curva} mostra esse histórico. É interessante observar que com o decaimento dos fatores de exploração e de aprendizagem, após 100000 partidas ambos eram $\epsilon \approx 0.074074$ e $\alpha \approx 0.056604$, ou seja, o agente já executava na maior parte dos ciclos a política aprendida. Para cada jogo foi feita a média entre os 3 retornos observados em cada um dos treinamentos.

\begin{figure}[H]
	\includegraphics[width=0.93\linewidth]{figs/curva-qtabular.jpg}
	\centering
	\caption{Curva de aprendizado do agente com ações puras. }
	\label{fig:single-agent-curva}
\end{figure}

\begin{figure}[H]
	\includegraphics[width=0.93\linewidth]{figs/curvalonga-qtabular.jpg}
	\centering
	\caption{Curva de aprendizado para treinamento longo.}
	\label{fig:single-agent-curvalonga}
\end{figure}

Além disso, assim como na Seção \ref{sec:behaviors-tabular}, foi executado um treinamento de 200000 partidas a fim de observar a aprendizagem por um período mais longo. Na Figura \ref{fig:single-agent-curvalonga} observa-se que o agente continua melhorando seu desempenho até próximo do fim do treinamento, o que pode ser indício de que o aprendizado com ações puras de fato permite mais flexibilidade na política aprendida.

Apesar da grande quantidade de experiência a que o agente teve acesso, nota-se na Figura \ref{fig:single-agent-curvalonga} que o crescimento de seu desempenho é bastante limitado, sequer atingindo a média de 1 gol por partida. Isso é um indicativo do altíssimo custo computacional de soluções \textit{end-to-end} como a utilizada no experimento.

O capítulo seguinte relata as conclusões acerca desse trabalho, sua contribuição para a área e discorre sobre possíveis trabalhos para continuação do desenvolvimento da plataforma, especialmente dentro do contexto da Universidade de Brasília.

% \section{Agentes Concorrentes}

% Após validação do sistema com agente único, é interessante experimentar com treinamento adversarial de apenas 2 jogadores em formato um-contra-um. A intenção dessa etapa é experimentar com o sistema o caso adversarial, no qual há um ou mais agentes com objetivo oposto ao do agente sendo treinado.
 
% \section{Múltiplos Agentes}

% Após validar os casos de agente único e de agentes concorrentes, propõe-se um treinamento completo em jogos 11 contra 11. O objetivo é, ao final do processo, termos um time capaz de jogar contra os principais times da atualidade na categoria RoboCup Soccer Simulation 2D.

% Para isso, os agentes devem ser capazes de cooperar e reagir aos movimentos da equipe oposta a fim de marcar gols e evitar os gols do adversário.



\begin{comment}
Conclusões
\end{comment}

\chapter{Conclusões}
\label{chap:Conclusoes}

Este projeto teve como objetivo a implementação de uma plataforma para desenvolvimento de aprendizagem por reforço em futebol de robôs, mais especificamente na categoria RoboCup Soccer Simulation 2D, e a utilização de técnicas de RL para realizar treinamentos de seleção de comportamentos e ações puras para maximizar o número de gols feitos por um agente.

Ao pesquisar sobre a comunidade e equipes participantes das edições nacionais da competição, notou-se que a biblioteca de interfaceamento com o servidor \textit{librcsc} e o time base \textit{agent2d} - ambos desenvolvidos no Japão por acadêmicos da equipe \textit{HELIOS} - são amplamente utilizados. Apesar de ser uma plataforma bastante completa, sua dominância acaba limitando a inovação na categoria, mantendo as equipes dentro da arquitetura de solução proposta no \textit{agent2d}.

A plataforma desenvolvida no decorrer deste trabalho ajuda a modernizar e diversificar a base de código utilizada pelas equipes. A plataforma desenvolvida encontra-se disponível em um repositório do GitHub: \textit{rcssggb/ggb-lib}, para ser utilizada livremente. A plataforma se conecta via protocolo UDP ao servidor que executa a partida e abstrai a codificação e decodificação de mensagens enviadas ao servidor. Algoritmos de RL implementados no decorrer do trabalho validaram o funcionamento da plataforma.

Foram realizados três treinamentos distintos e seus resultados foram analisados conforme o retorno alcançado pelo agente. Inicialmente, o algoritmo Sarsa com comportamentos pré-programados e aproximador de funções mostrou um retorno crescente com o passar dos episódios, porém o custo computacional do treinamento mostrou-se muito alto, tornando difícil a testagem de soluções desse tipo.

A fim de aumentar o número de amostras, o estado foi discretizado e implementou-se o algoritmo \textit{Q-Learning} duplo com os mesmos comportamentos e utilizando um método tabular. O número de amostras possíveis de serem coletadas aumentou consideravelmente e o treinamento se mostrou mais estável, além de obter retornos mais altos.

Por fim, foi realizado um treinamento com ações puras, permitindo que o agente aprendesse estratégias por si só. O treinamento demonstrou retornos menores que o anterior, porém a aprendizagem pareceu continuar por mais tempo, o que sugere que abordagens \textit{end-to-end} são capazes de aproveitar melhor quantidades maiores de amostras.

\section{Trabalhos Futuros}

A plataforma desenvolvida abre muitas possibilidades para pesquisa e desenvolvimento dentro do contexto da RoboCup Soccer Simulation 2D. É possível estudar diversas técnicas de RL a fim de cumprir a proposta da competição: desenvolver um time de futebol completo com agentes capazes de cooperar entre si e disputar contra outras equipes participantes.

Apesar da abordagem simplificada feita neste trabalho, o formato oficial da competição envolve um ambiente multiagente e adversarial que introduz diversos desafios ao aprendizado uma vez que o ambiente se torna mais dinâmico e, com isso, mais imprevisível.

Além disso, a família de técnicas de RL ideais para um cenário como a RCSS é a família métodos de gradientes de políticas, ou seja, o agente busca estimar a política em si e não a função de valor de ação $Q$. Com essas técnicas, é possível parametrizar a política de modo a cobrir todo o espaço de ações e não apenas um subconjunto discreto dele.

Por fim, em um cenário competitivo é importante manter a plataforma atualizada e capaz de abstrair todas as funcionalidades do servidor. Uma das funcionalidades faltantes, por exemplo, é a da investida, também conhecida coloquialmente como "carrinho".

As equipes acadêmicas são um ambiente propício para o aprendizado prático de diversas técnicas aprendidas durante a graduação e a Universidade de Brasília tem uma longa tradição de equipes participantes em diversas categorias de robótica, incluindo outras categorias de futebol de robôs. A organização de uma equipe para desenvolvimento utilizando a plataforma a fim de participar da RCSS criaria diversas oportunidades de aprendizado e pesquisa.

% multiagente
% expandir as funcionalidades da plataforma
% adversarial
% algoritmos que tratem ações contínuas

\begin{comment}
Bibliografia
\end{comment}


\renewcommand{\bibname}{REFERÊNCIAS BIBLIOGRÁFICAS}
\addcontentsline{toc}{chapter}{REFERÊNCIAS BIBLIOGRÁFICAS}

\bibliographystyle{abnt-num}
\bibliography{bibliography}


\begin{comment}
Anexos
\end{comment}


% \anexos
\makeatletter
% não retirar estes comandos
\renewcommand{\@makechapterhead}[1]{%
  {\parindent \z@ \raggedleft \setfontarial\bfseries
\LARGE \thechapter. \space\space
\uppercase{#1}\par
\vskip 40\p@
}
}
\makeatother

\begin{comment}
Anexo I: Descrição do CD
\end{comment}


% \include{anexo_CD}

% \refstepcounter{noAnexo}

\begin{comment}
Anexo II: Programas Utilizados
\end{comment}


% \include{anexo_Codigos}

% \refstepcounter{noAnexo}

\begin{comment}
Acrescente mais anexos conforme julgar necessário.
\end{comment}

\end{document}
